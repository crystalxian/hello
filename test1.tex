\documentclass{ctexart}
\usepackage{amsmath}
\usepackage{amssymb}
\usepackage{graphicx}
\usepackage{subfigure}
\usepackage[colorlinks,linkcolor=blue,citecolor=green,anchorcolor=blue]{hyperref}
\author{董泽贤\\2013011182}
\title{数字图像处理作业}
\begin{document}
\makeatletter %使\section中的内容左对齐
\renewcommand{\section}{\@startsection{section}{1}{0mm}
  {-\baselineskip}{0.5\baselineskip}{\bf\leftline}}
\makeatother
\maketitle
\section {题目2-1}
\begin{flushleft}
平移变换矩阵T~~(X方向平移量是Y方向平移量的一半)
\[\ \boldsymbol{T} =
\left[
\begin{array}{ccc}
1 & 0 & T_x \\
0 & 1 & 2T_x \\
0 & 0 & 1
\end{array}
\right]
\]
尺度变换矩阵S~~(X方向放缩量是Y方向放缩量的两倍)
\[\ \boldsymbol{S} =
\left[
\begin{array}{ccc}
2S_y & 0 & 0 \\
0 & S_y & 0 \\
0 & 0 & 1
\end{array}
\right]
\]
(5,10)先经过平移变换后尺度变换结果
\[
\left[
\begin{array}{c}
10S_y+2S_yT_x\\
10S_y+2S_yT_x\\
1
\end{array}
\right]
=
\left[
\begin{array}{ccc}
2S_y & 0 & 0 \\
0 & S_y & 0 \\
0 & 0 & 1
\end{array}
\right]
%
\left[
\begin{array}{ccc}
1 & 0 & T_x \\
0 & 1 & 2T_x \\
0 & 0 & 1
\end{array}
\right]
%
\left[
\begin{array}{c}
5\\
10\\
1
\end{array}
\right]
\]
(5,10)先经过尺度变换后平移变换结果
\[
\left[
\begin{array}{c}
10S_y+T_x\\
10S_y+2T_x\\
1
\end{array}
\right]
=
\left[
\begin{array}{ccc}
1 & 0 & T_x \\
0 & 1 & 2T_x \\
0 & 0 & 1
\end{array}
\right]
%
\left[
\begin{array}{ccc}
2S_y & 0 & 0 \\
0 & S_y & 0 \\
0 & 0 & 1
\end{array}
\right]
%
\left[
\begin{array}{c}
5\\
10\\
1
\end{array}
\right]
\]
通过上面的计算结果可以看出,变换级联的效果与变换的次序有关,所以变换矩阵的次序不能随意更换。
%\end{flushleft}
\section {题目3-11}
%\begin{flushleft}
(1)~~该算法的效果特点\\
根据算法描述,可以看出其能平滑图像,模糊轮廓,有空域平滑滤波的效果

\begin{figure}[h]
\centering
\subfigure[原图]{
\label{Fig.sub.0}
\includegraphics[width=0.3\textwidth]{Lena0.eps}}
\subfigure[一次操作]{
\label{Fig.sub.1}
\includegraphics[width=0.3\textwidth]{lena1.eps}}
\subfigure[五次操作]{
\label{Fig.sub.2}
\includegraphics[width=0.3\textwidth]{lena3.eps}}
\caption{Matlab编程实现结果}
\label{Fig.lable}
\end{figure}
Matlab编程印证了其平滑滤波的效果
\href{https://github.com/crystalxian/hello.git}{代码}\\

~\\
(2)~~反复利用该算法,得到的效果\\
~~~~反复利用本算法,可以抹平轮廓
\begin{figure}[h]
\centering
\subfigure[原图]{
\label{Fig.sub.0}
\includegraphics[width=0.4\textwidth]{Lena0.eps}}
\subfigure[二十次操作]{
\label{Fig.sub.3}
\includegraphics[width=0.4\textwidth]{lena4.eps}}
\caption{反复利用该算法结果}
\label{Fig.lable1}
\end{figure}
\end{flushleft}
\newpage
\end{document}
